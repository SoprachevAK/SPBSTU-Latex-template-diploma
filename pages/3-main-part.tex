\chapter{Название главы}
  \section{Названия секции}
    \subsubsection{Название подсекции}
      \paragraph{Название параграфа}
        \subparagraph{Название подпараграфа}
          \lipsum[2]\cite{BibExampleRU}
          \begin{center}
            \begin{xltabular}{\linewidth}{|l|X|}
              \caption{Long table caption.\label{long}}                                                                                                    \\
              \hline
              Столбик1  & Столбик2    \\
              \hline
              слово     & Слово       \\
              \hline
            \end{xltabular}
          \end{center}


\chapter{Название главы}
  \section{Названия секции}
    \subsubsection{Название подсекции}
      \paragraph{Название Параграфа}
        \lipsum[2]\cite{BibExampleRU}
        Как видно на \refImg{img:test} или на \refAppendix{appendix:example}
        \fig[Длинное название картинки из примера][0.5][img:test]{assets/example.drawio.png}

        \begin{center}
          \begin{xltabular}{\linewidth}{|l|X|}
            \caption{Long table caption.\label{long}}                                                                                                    \\
            \hline
            Столбик1  & Столбик2    \\
            \hline
            слово     & Слово       \\
            \hline
          \end{xltabular}
        \end{center}

        \begin{lstlisting}[caption={Название листинга}]
begin
  print('Hellow world!')
end
        \end{lstlisting}

        \begin{itemize}
          \item item1
          \item item2
        \end{itemize}

        \begin{enumerate}
          \item item1
          \item item2
        \end{enumerate}
